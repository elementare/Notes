gdocumentclass{article}

% ----------------------
% Language and Encoding
% ----------------------
\usepackage{fontspec}
\usepackage{luatexja}
\usepackage{luatexja-fontspec}
\setmainjfont{Noto Serif CJK JP}

% ----------------------
% Mathematics (Legacy-Compatible)
% ----------------------
\usepackage{amsmath, amssymb, amsthm, mathtools, mathpartir}
\usepackage{tikz}
\usepackage{tikz-cd}

% ----------------------
% General Formatting
% ----------------------
\usepackage{graphicx}
\usepackage{geometry}
\usepackage{indentfirst}
\usepackage{hyperref}
\usepackage{enumerate}
\usepackage{tabularx}
\usepackage{cancel}
\usepackage{comment}
\usepackage{hyphenat}
\usepackage{ellipsis}
\usepackage{paracol}
\usepackage{xcolor}
\usepackage[most]{tcolorbox}

% ----------------------
% Logic (Proofs)
% ----------------------
\usepackage{fitch}
\usepackage{proof}

% ----------------------
% Math Shortcuts
% ----------------------
\newcommand*\Implies{\mathrel{\Rightarrow}}
\renewcommand*\implies{\mathrel{\rightarrow}}
\newcommand*\Iff{\mathrel{\Leftrightarrow}}
\renewcommand*\iff{\mathrel{\leftrightarrow}}
\renewcommand*\land{\mathrel{\&}}
\newcommand*\ex{\mathrel{\downarrow}}

\DeclareMathOperator{\sgn}{sgn}
\DeclareMathOperator{\lcm}{lcm}
\DeclareMathOperator{\mmc}{mmc}
\DeclareMathOperator{\mdc}{mdc}
\DeclareMathOperator{\erf}{erf}
\DeclareMathOperator{\ifc}{if}
\let\Re\relax
\DeclareMathOperator{\Re}{Re}
\let\Im\relax
\DeclareMathOperator{\Im}{Im}

\renewcommand*\d{\mathrm{d}}

% Sets
\newcommand*\N{\mathbb{N}}
\newcommand*\Z{\mathbb{Z}}
\newcommand*\Q{\mathbb{Q}}
\newcommand*\R{\mathbb{R}}
\newcommand*\C{\mathbb{C}}
\newcommand*\F{\mathbb{F}}

% Other math utils
\newcommand*{\mbb}[1]{\mathbb{#1}}
\newcommand*{\mcl}[1]{\mathcal{#1}}
\newcommand*{\mfr}[1]{\mathfrak{#1}}
\newcommand*{\floor}[1]{\left\lfloor#1\right\rfloor}
\newcommand*{\ceil}[1]{\left\lceil#1\right\rceil}
\renewcommand*\O{\mathcal{O}}
\renewcommand*\emptyset{\varnothing}
\newcommand*\df{\mathrm{df}}

\everymath{\displaystyle}

% ----------------------
% Links
% ----------------------
\hypersetup{
    bookmarksnumbered=true,
    colorlinks=true,
    linkcolor=black,
    citecolor=black,
    urlcolor=blue,
}

% ----------------------
% Playing card symbols
% ----------------------
\DeclareSymbolFont{extraup}{U}{zavm}{m}{n}
\DeclareMathSymbol{\spades}{\mathalpha}{extraup}{81}
\DeclareMathSymbol{\clubs}{\mathalpha}{extraup}{84}
\DeclareMathSymbol{\hearts}{\mathalpha}{extraup}{86}
\DeclareMathSymbol{\diamonds}{\mathalpha}{extraup}{87}

% ----------------------
% Fix \pmod*
% ----------------------
\makeatletter
\let\@@pmod\pmod
\DeclareRobustCommand{\pmod}{\@ifstar\@pmods\@@pmod}
\def\@pmods#1{\mkern4mu({\operator@font mod}\mkern 6mu#1)}
\makeatother

% ----------------------
% Better \widebar
% ----------------------
\makeatletter
% (Seu código customizado para widebar pode entrar aqui se quiser manter. Omiti por brevidade.)
\makeatother

% ----------------------
% Documento
% ----------------------
\title{Mathematica Collectanea}
\author{eyeS}
\date{March 2025}

% Vintage Parchment Theme
% Inspired by the soft yellow tones of old paper.

\usepackage{xcolor}
\usepackage{pagecolor}

% Soft parchment-like background
\definecolor{parchment}{RGB}{250, 245, 229}   % light yellow-beige
\definecolor{textblack}{RGB}{30, 30, 30}      % dark grayish text

\pagecolor{parchment}
\color{textblack}


\begin{document}

\begin{titlepage}
    \centering
    \vspace*{3cm}
    {\Huge \bfseries Biology Collectanea \par}
    \vspace{0.5cm}
    {\large A Personal Compendium of Biology Notes\par}
    \vfill
    {\LARGE \bfseries eyeS\par}
    \vspace{0.5cm}
    {\Large \textsc{March 2025}}
\end{titlepage}

\tableofcontents
\clearpage

% Notes content
\part{Category Theory}
\section*{Basic Definitions}


Category theory deals with the generalization of structures, it starts with the observation that many properties of mathematical systems can be unified and simplified by a presentation with diagram of arrows [pg 1]. Each arrow $f: X \to Y$ represents a function; that is, a set $X$, a set $Y$, and a rule $x \mapsto fx$ which assigns to each element $x \in X$ an element $fx \in Y$; whenever possible we write $fx$ and not $f(x)$. A typical diagram of sets and functions looks like: 
\begin{center}
\begin{tikzcd}
  & Y \arrow[dr, "g"'] & \\
  X \arrow[rr, "h"'] \arrow[ur, "f"'] & & Z
\end{tikzcd}
\end{center}

This diagram is commutative when $h$ is $h = g \circ f$, where $g \circ f$ is the usual composite function $g \circ f: X \to Z$. The same diagram can be applied to other mathematical structures, like "category" of all spaces, the letters $X$, $Y$ and $Z$ represents topological spaces, while $f, g, h$ stand for continuous maps. The same can be used for "category" of all groups e.g.


Many properties of mathematical construction can be represented by the universal properties of diagram. Consider the cartesian product $X \times Y$ of two sets. The projections $<x,y> \mapsto x, <x, y> \mapsto y$ of the product are function $p: X \times Y \rightarrow X, q: X \times Y \rightarrow Y$. Any function $h: W \to X \times Y$ from a third set $W$ is uniquely determined by its composites $p \circ h$ and $q \circ h$. Conversly, given W and two functions $f, g$ as in the diagram below, there is a unique function $h$ which makes the diagram commute, namely, $hw = <fw, gw>$ for each $w \in W$:

\begin{center}
\begin{tikzcd}
   & \arrow[dl, "f"'] W \arrow[d, "h"'] \arrow[dr, "g"'] &\\
  X & \arrow[l, "p"'] X \times Y \arrow[r, "q"'] & Y
\end{tikzcd}
\end{center}

Thus, given $X$ and $Y$, $<p, q>$ is "universal" among pairs of functions from some set to $X$ and $Y$, because any other such pair $<f, g>$ factors uniquely (via h) through the pair $<p ,q>$

We define \textbf{Hom-set} from $X$ to $Y$ as $hom(X, Y) = \{f; f: X \to Y\}$. The "hom" comes from \textbf{homomorphism}, a general term for structure-preserving maps in category theory. In the set-theory, every function is a homomorphism because we don't care about the extra structure (like in groups or topological spaces)



\begin{examplebox}[title={Universal Property of the Product}]
Let 
\[
W = \{1, 2\}, \quad X = \{a, b\}, \quad Y = \{x, y\}.
\]
Let the projections be 
\[
p: X \times Y \to X, \quad p(a, x) = a, \quad q: X \times Y \to Y, \quad q(a, x) = x.
\]

We define two functions:
\[
f: W \to X, \quad f(1) = a,\ f(2) = b; \qquad
g: W \to Y, \quad g(1) = x,\ g(2) = y.
\]

Then there exists a unique function:
\[
h: W \to X \times Y, \quad h(w) = (f(w), g(w)),
\]
such that the following diagram commutes:

\begin{center}
\begin{tikzcd}
   & \arrow[dl, "f"'] W \arrow[d, "h"'] \arrow[dr, "g"] &\\
  X & \arrow[l, "p"] X \times Y \arrow[r, "q"'] & Y
\end{tikzcd}
\end{center}

That is, \( p \circ h = f \) and \( q \circ h = g \). For example:
\[
h(1) = (a, x), \quad h(2) = (b, y)
\]

We could also define a different pair:
\[
f': W \to X, \quad f'(1) = a,\ f'(2) = a; \qquad
g': W \to Y, \quad g'(1) = x,\ g'(2) = y,
\]
yielding another unique map \( h': W \to X \times Y \) given by \( h'(w) = (f'(w), g'(w)) \), which also factors uniquely through \( \langle p, q \rangle \).

\[
h'(1) = (a, x), \quad h'(2) = (a, y)
\]

This illustrates that every pair of functions \( f, g \) from a common domain factors uniquely through the product projections \( p, q \), verifying the universal property.

\end{examplebox}


\end{document}

