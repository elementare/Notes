\documentclass[12pt, a4paper]{article}
% ==========================
% UNIFIED TEX PREAMBLE
% ==========================

% ----------------------
% Coding and Language
% ----------------------


\usepackage[main=brazilian]{babel}

% ----------------------
% Typography and mathematics
% ----------------------
\usepackage{amsmath}
\usepackage{amssymb}
\usepackage{amsthm}
\usepackage{mathtools}
\usepackage{mathpartir}


% --------------------
% Symbol and utility packages
% --------------------
\usepackage{cancel}
\usepackage{textcomp}
\usepackage[mathscr]{euscript}
\usepackage[nointegrals]{wasysym}

% --------------------
% Extras
% --------------------
\usepackage{physics}
\usepackage{tikz}
\usepackage{tikz-cd}
\usetikzlibrary{decorations.markings}
\usetikzlibrary{calc}
\usepackage{graphicx}
\usepackage{geometry}
\usepackage{fix-cm}
\usepackage{esvect}
\usepackage{nicematrix}
\usepackage{CJKutf8}
\usepackage{esint}
\usepackage{siunitx}
\usepackage{pgfplots}
\usepackage{circuitikz}
\usepackage{wrapfig}
% ----------------------
% Utilities and formatation
% ----------------------
\usepackage{mathrsfs}
\usepackage{indentfirst}
\usepackage{hyperref}
\usepackage{enumerate}
\usepackage{tabularx}
\usepackage{cancel}
\usepackage{comment}
\usepackage{csquotes}
\usepackage{hyphenat}
\usepackage{ellipsis}
\usepackage{paracol}
\usepackage[T1]{tipa}
\usepackage{xcolor}
\usepackage{fullwidth}
% ----------------------
% Logic and mathematics
% ----------------------
\usepackage{fitch}
\usepackage{proof}

\renewcommand*\land{\mathrel{\&}}
\newcommand*\Implies{\mathrel{\Rightarrow}}
\renewcommand*\implies{\mathrel{\rightarrow}}
\newcommand*\Iff{\mathrel{\Leftrightarrow}}
\renewcommand*\iff{\mathrel{\leftrightarrow}}
\newcommand*\ex{\mathrel{\downarrow}}

% ----------------------
% Customized math operators
% ----------------------
\DeclareMathOperator{\sgn}{sgn}
\DeclareMathOperator{\lcm}{lcm}
\DeclareMathOperator{\mmc}{mmc}
\DeclareMathOperator{\mdc}{mdc}
\DeclareMathOperator{\ifc}{if}
\let\Re\relax
\DeclareMathOperator{\Re}{Re}
\let\Im\relax
\DeclareMathOperator{\Im}{Im}

\renewcommand*\d{\mathrm{d}}

% Other math keybinds
\newcommand*{\mbb}[1]{\mathbb{#1}}
\newcommand*{\mcl}[1]{\mathcal{#1}}
\newcommand*{\mfr}[1]{\mathfrak{#1}}
\newcommand*{\func}[3]{#1:#2\to#3}
\newcommand{\vfunc}[5]{\func{#1}{#2}{#3},\quad#4\longmapsto#5}
\newcommand*{\floor}[1]{\left\lfloor#1\right\rfloor}
\newcommand*{\ceil}[1]{\left\lceil#1\right\rceil}


\newcommand*\df{\mathrm{df}}

\newcommand*\N{\mathbb{N}}
\newcommand*\Z{\mathbb{Z}}
\newcommand*\Q{\mathbb{Q}}
\newcommand*\R{\mathbb{R}}
\newcommand*\C{\mathbb{C}}
\newcommand*\F{\mathbb{F}}

\renewcommand*\O{\mathcal{O}}
\renewcommand*\emptyset{\varnothing}

% Some standard theorem definitions
\newtheorem{Theorem}{Theorem}
\newtheorem{Proposition}{Theorem}
\newtheorem{Lemma}[Theorem]{Lemma}
\newtheorem{Corollary}[Theorem]{Corollary}

\theoremstyle{definition}
\newtheorem{Definition}[Theorem]{Definition}


% ----------------------
% Always use math display
% ----------------------
\everymath{\displaystyle}

% ----------------------
% Configuração dos links
% ----------------------
\hypersetup{
    bookmarksnumbered=true,
    colorlinks=true,
    linkcolor=black,
    citecolor=black,
    urlcolor=blue,
}

% ----------------------
% Unicode Symbols (♠ ♣ ♥ ♦)
% ----------------------
\DeclareSymbolFont{extraup}{U}{zavm}{m}{n}
\DeclareMathSymbol{\spades}{\mathalpha}{extraup}{81}
\DeclareMathSymbol{\clubs}{\mathalpha}{extraup}{84}
\DeclareMathSymbol{\hearts}{\mathalpha}{extraup}{86}
\DeclareMathSymbol{\diamonds}{\mathalpha}{extraup}{87}

% ----------------------
% Correção do comportamento do \pmod*
% ----------------------
\makeatletter
\let\@@pmod\pmod
\DeclareRobustCommand{\pmod}{\@ifstar\@pmods\@@pmod}
\def\@pmods#1{\mkern4mu({\operator@font mod}\mkern 6mu#1)}
\makeatother

% ----------------------
% Melhor Overline (\widebar)
% ----------------------
\makeatletter
\let\save@mathaccent\mathaccent
\newcommand*\if@single[3]{%
  \setbox0\hbox{${\mathaccent"0362{#1}}^H$}%
  \setbox2\hbox{${\mathaccent"0362{\kern0pt#1}}^H$}%
  \ifdim\ht0=\ht2 #3\else #2\fi
}
\newcommand*\rel@kern[1]{\kern#1\dimexpr\macc@kerna}
\newcommand*\wideaccent[2]{\@ifnextchar^{{\wide@accent{#1}{#2}{0}}}{\wide@accent{#1}{#2}{1}}}
\newcommand*\wide@accent[3]{\if@single{#2}{\wide@accent@{#1}{#2}{#3}{1}}{\wide@accent@{#1}{#2}{#3}{2}}}
\newcommand*\wide@accent@[4]{%
  \begingroup
  \def\mathaccent##1##2{%
    \let\mathaccent\save@mathaccent
    \if#42 \let\macc@nucleus\first@char \fi
    \setbox\z@\hbox{$\macc@style{\macc@nucleus}_{}$}%
    \setbox\tw@\hbox{$\macc@style{\macc@nucleus}{}_{}$}%
    \dimen@\wd\tw@
    \advance\dimen@-\wd\z@
    \divide\dimen@ 3
    \@tempdima\wd\tw@
    \advance\@tempdima-\scriptspace
    \divide\@tempdima 10
    \advance\dimen@-\@tempdima
    \ifdim\dimen@>\z@ \dimen@0pt\fi
    \rel@kern{0.6}\kern-\dimen@
    \if#41
      #1{\rel@kern{-0.6}\kern\dimen@\macc@nucleus\rel@kern{0.4}\kern\dimen@}%
      \advance\dimen@0.4\dimexpr\macc@kerna
      \let\final@kern#3%
      \ifdim\dimen@<\z@ \let\final@kern1\fi
      \if\final@kern1 \kern-\dimen@\fi
    \else
      #1{\rel@kern{-0.6}\kern\dimen@#2}%
    \fi
  }%
  \macc@depth\@ne
  \let\math@bgroup\@empty \let\math@egroup\macc@set@skewchar
  \mathsurround\z@ \frozen@everymath{\mathgroup\macc@group\relax}%
  \macc@set@skewchar\relax
  \let\mathaccentV\macc@nested@a
  \if#41
    \macc@nested@a\relax111{#2}%
  \else
    \def\gobble@till@marker##1\endmarker{}%
    \futurelet\first@char\gobble@till@marker#2\endmarker
    \ifcat\noexpand\first@char A\else
      \def\first@char{}%
    \fi
    \macc@nested@a\relax111{\first@char}%
  \fi
  \endgroup
}
\newcommand\widebar{\wideaccent\overline}
\makeatother


% Vintage Parchment Theme
% Inspired by the soft yellow tones of old paper.

\usepackage{xcolor}
\usepackage{pagecolor}

% Soft parchment-like background
\definecolor{parchment}{RGB}{250, 245, 229}   % light yellow-beige
\definecolor{textblack}{RGB}{30, 30, 30}      % dark grayish text

\pagecolor{parchment}
\color{textblack}
\definecolor{examplebg}{RGB}{255, 252, 240}  % slightly lighter parchment
\definecolor{exampleborder}{RGB}{200, 180, 120} % soft gold-brown

\tcbset{
  myexamplebox/.style={
    enhanced,
    colback=examplebg,
    colframe=exampleborder,
    coltitle=textblack,
    boxrule=0.5pt,
    arc=3mm,
    sharp corners=south,
    fonttitle=\bfseries,
    title={Example},
    left=6pt,
    right=6pt,
    top=6pt,
    bottom=6pt,
    #1
  }
}

\newtcolorbox{examplebox}[1][]{myexamplebox,#1}



\title{Relatório Espaços Métricos}
\author{Carlos Daniel}
\date{2025-03-31}

\begin{document}

\maketitle
\clearpage

% Notes content
\setcounter{section}{1}
\section*{Basic Definitions}

Given a statement of the form $P \implies Q$, its \textbf{contrapositive} is defined to be the statement $\neg Q \implies \neg P$. For example, the contrapositive of the statement:

\begin{center}
  $x > 0 \longrightarrow \neg x^3 \neq 0$
\end{center}

is the statement

\begin{center}
  $x^3 = 0 \longrightarrow \neg x > 0$
\end{center}.

It's trivial to demonstrate the $P \implies Q \iff \neg P \implies Q$. Another useful definitionn is the \textbf{converse} of some statement. Given $P \implies Q$, the converse of this statement is $Q \implies P$.

\begin{definition}[1]
  A \textbf{rule of assignment} is a subset $r$ of the cartesian product $C \times D$ of two sets, having the property that each element of $C$ appears as the first coordinate of \textit{at most one} ordered pair belonging to $r$. Thus, a subset $r$ of $C \times D$ is a rule assignment if: 
  \begin{equation}
     [(c, d) \in r \text{ and } (c,d') \in r] \implies [d = d']
  \end{equation}
\end{definition}

\begin{definition}
  A \textbf{function} $f$ is a rule of assignment r, together with a set $B$ that contains the image of $r$. The domain of A of the rule $r$ is also called the \textbf{domain} of the function $f$; the image set of $r$ is also called the \textbf{image set} of $f$; and the set $B$ is called the \textbf{range} of $f$. If $f$ is a function having domain $A$ and range $B$, we express the fact by writing
  \begin{displaymath}
    f: A \longrightarrow B
  \end{displaymath}
  
\end{definition}
Formally, if $r$ is the rule of the function $f$, then $f(a)$ denotes the unique element of $B$ such that $(a, f(a)) \in r$.
\begin{definition}
  If $f:A \to B$ and if $A_0$ is a subset of $A$, we define the \textbf{restriction} of $f$ to $A_0$ to be the function mapping $A_0$ into $B$ whose rules is

  \begin{equation}
    \{(a, f(a) | a \in A_0)\}.
    \label{eq:restriction function}
  \end{equation}
It is denoted by $f|A_0$, which is read "$f$ restricted to $A_0$" 
\end{definition}

The book stats with a really useful definitions (but I'm lazy, so I'll do it quickly). A function $f: A \to B$ is said to be \textbf{injective} (or \textbf{one-to-one}) if:
\begin{equation}
  [f(a) = f(a')] \implies [a = a']
\end{equation}

It is said to be \textbf{surjective} (or $f$ is a map \textbf{onto} B) if:
\begin{equation}
  [b \in B] \implies [\exists a \in A; b = f(a)]
\end{equation}

If $f$ is both injective and surjective, it is said to be \textbf{bijective} (or is called a \textbf{one-to-one correspondence})

Injectivty of $f$ depends only on the rule of $f$; surjective depends on the range of $f$ as well. It's valid that composition of functions with same "type" has the same "type" (injective, surjective)
If $f$ is bijective, there exists a function from $B$ to $A$ called the \textbf{inverse} of $f$. It is denoted by $f^{-1}$ and is defined by letting $f^{-1}(b)$ be the unique element $a \in A$ for which $f(a) = b$. It's easy to see if $f$ is bijective, then $f^{-1}$ is also bijective.

\begin{lemma}
  Let $f: A \to B$. If there are functions $g: B \to A$ and $h: B \to A$ such that $g(f(a)) = a$ for every $a \in A$ and $f(h(b)) = b$ for every $b \in B$, the $f$ is bijective and $g = h = f^{-1}$ 
\end{lemma}

\begin{proof}
If we have a function $g$ such that $g(f(a)) = a, \forall a \in A$, then $f$ must to be injective. Suppose $m,n in A$ which $g(f(m)) = m \land g(f(n)) = n$, if $f(m) = f(n)$, then $g(f(n)) = g(f(m))$ which means $m = n$. Conversely, if $m = n$, then $g(f(n)) = g(f(m))$, which means $f(m) = f(n)$. So $f(m) = f(n) \iff m = n$, that proves that $f$ is injective. If we have a function $h$ such that $f(h(b)) = b, \forall b \in B$, then $f$ must to be surjective.  
\end{proof}



\end{document}

